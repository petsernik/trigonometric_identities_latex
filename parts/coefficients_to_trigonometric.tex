\subsection*{Коэффициенты как синусы и косинусы с множителем}

\begin{align*}
  a \sin x + b \cos x &= \sqrt{a^2+b^2} \left(\frac{a}{\sqrt{a^2+b^2}} \sin x + \frac{b}{\sqrt{a^2+b^2}}\cos x\right) \\ 
  &= \sqrt{a^2+b^2} (\sin \alpha \sin x + \cos \alpha \cos x) = \sqrt{a^2+b^2} \cos(x - \alpha) \\
  &= \sqrt{a^2+b^2} (\cos \beta \sin x + \sin \beta \cos x) = \sqrt{a^2+b^2} \sin(x + \beta) \\
  &\text{где } \alpha=\arcsin \frac{a}{\sqrt{a^2+b^2}},\ \beta = \arcsin \frac{b}{\sqrt{a^2+b^2}}
\end{align*}
