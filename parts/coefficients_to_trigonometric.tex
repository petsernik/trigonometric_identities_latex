\subsection*{Коэффициенты как синусы и косинусы с множителем}

\begin{align*}
  a \sin x + b \cos x &= \sqrt{a^2+b^2} \left(\frac{a}{\sqrt{a^2+b^2}} \sin x + \frac{b}{\sqrt{a^2+b^2}}\cos x\right) \\ 
  &= \sqrt{a^2+b^2} (\cos \varphi \sin x + \sin \varphi \cos x) = \sqrt{a^2+b^2} \sin(x + \varphi) \\
  &= \sqrt{a^2+b^2} (\sin \tau \sin x + \cos \tau \cos x) = \sqrt{a^2+b^2} \cos(x - \tau)
\end{align*}

\textit{Примечание.} данный подход напоминает переход в полярные координаты, где точка $(a,b)$(в евклидовых координатах) задана как $(r,\varphi)$(в полярных координатах) с $r=\sqrt{a^2+b^2}, \varphi = \pm \arccos\frac{a}{r}$, выбор знака $\varphi$ зависит от знака $b$ (они должны совпадать по знаку) --- это актуально и для применения формулы выше.